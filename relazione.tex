% !TEX TS-program = pdflatex
% !TEX encoding = UTF-8 Unicode

% This is a simple template for a LaTeX document using the "article" class.
% See "book", "report", "letter" for other types of document.

\documentclass[11pt]{article} % use larger type; default would be 10pt

\usepackage[utf8]{inputenc} % set input encoding (not needed with XeLaTeX)
\usepackage[italian]{babel} 
%%% Examples of Article customizations
% These packages are optional, depending whether you want the features they provide.
% See the LaTeX Companion or other references for full information.

%%% PAGE DIMENSIONS
\usepackage{geometry} % to change the page dimensions
\geometry{a4paper} % or letterpaper (US) or a5paper or....
\geometry{margin=1.2in} % for example, change the margins to 2 inches all round
% \geometry{landscape} % set up the page for landscape
%   read geometry.pdf for detailed page layout information
\usepackage{float}
\usepackage{mathabx}
\usepackage{adforn}
\usepackage{rotating}
\usepackage{graphicx} % support the \includegraphics command and options

% \usepackage[parfill]{parskip} % Activate to begin paragraphs with an empty line rather than an indent

%%% PACKAGES
\usepackage{booktabs} % for much better looking tables
\usepackage{array} % for better arrays (eg matrices) in maths
\usepackage{paralist} % very flexible & customisable lists (eg. enumerate/itemize, etc.)
\usepackage{verbatim} % adds environment for commenting out blocks of text & for better verbatim
\usepackage{subfig} % make it possible to include more than one captioned figure/table in a single float
\usepackage{wrapfig}
% These packages are all incorporated in the memoir class to one degree or another...

%%% HEADERS & FOOTERS
\usepackage{fancyhdr} % This should be set AFTER setting up the page geometry
\pagestyle{fancy} % options: empty , plain , fancy
\renewcommand{\headrulewidth}{0pt} % customise the layout...
\lhead{Fotoconducibilità}\chead{}\rhead{C.d.L. in Fisica}
\lfoot{}\cfoot{\thepage}\rfoot{}

%%% SECTION TITLE APPEARANCE
\usepackage{sectsty}
\allsectionsfont{\sffamily\mdseries\upshape} % (See the fntguide.pdf for font help)
% (This matches ConTeXt defaults)

%%% ToC (table of contents) APPEARANCE
\usepackage[nottoc,notlof,notlot]{tocbibind} % Put the bibliography in the ToC
\usepackage[titles,subfigure]{tocloft} % Alter the style of the Table of Contents
\renewcommand{\cftsecfont}{\rmfamily\mdseries\upshape}
\renewcommand{\cftsecpagefont}{\rmfamily\mdseries\upshape} % No bold!

%%% END Article customizations

%%% The "real" document content comes below...

\title{Fotoconducibilità}
\author{Michael Maguire, Leonardo Misuraca Giordano, Daniele Pani, Alberto Perro}
%\date{} % Activate to display a given date or no date (if empty),
         % otherwise the current date is printed 

\begin{document}
\maketitle
\newpage
\section*{Prefazione}
L'esperienza è volta alla misura della fotocorrente e della trasmittanza di diversi campioni di semiconduttore illuminati nello spettro NIR-visibile (1000 - 400 nm). Per selezionare la lunghezza d'onda è stato utilizzato un monocromatore a reticolo diffrattivo con risoluzione spettrale di 30 nm.
\section{Calibrazione del setup sperimentale}
\subsection{Calibrazione dello spettrometro}
Per la calibrazione dello spettrometro, che consiste nel trovare la funzione di conversione fra pixel e lunghezza d'onda, si è utilizzata una sorgente luminosa di spettro noto con picchi ben definiti (Miscela di Xeno - Argon) collegata in fibra ottica allo spettrometro.
Tramite l'utilizzo dell'apposito software, si sono acquisiti i dati impostando due diversi intervalli di integrazione (6 ms - 600$\mu$s) e 100 misure mediate, per evitare la saturazione dei picchi più luminosi. Per un'analisi approssimativa, si sono utilizzate le funzioni integrate del programma per stabilire i centri e i FWHM.\\Per la regressione, si è utilizzata una polinomiale di terzo grado (rosso), ma dato che i coefficienti di secondo e terzo grado estratti sono molto piccoli, si è svolto un fit con una retta, il quale però ha dato esito negativo.
\begin{figure}[h!]
\begin{center}
\includegraphics[width=350px]{img/cal_approx.pdf}
\caption{Calibrazione con i dati approssimati.}
\end{center}
\end{figure}
\\Successivamente analizzando i dati acquisiti dello spettro tramite ROOT, si sono svolti fit gaussiani per ogni picco, in modo da stabilire con maggiore precisione i parametri di conversione.\\
Dai risultati estratti si è quindi svolto nuovamente il fit di calibrazione, per utilizzare la miglior legge di conversione disponibile e verificare se l'errore della calibrazione approssimata risultasse significativo.
\begin{figure}[!h]
\centering
    \subfloat[]{{\includegraphics[width=7.9cm]{img/spec_fit.pdf} }}%
    \subfloat[]{{\includegraphics[width=7.9cm]{img/cal_gaus.pdf} }}%
\caption{Fit gaussiani dei picchi dello spettro (a) e calibrazione derivata (b).}
\end{figure}
\\
\subsection{Calibrazione del monocromatore interferenziale}
Stabilita la conversione pixel - lunghezza d'onda, si è effettuata la calibrazione del monocromatore lungo lo spettro che successivamente saremo andati ad utilizzare per le misure.
\begin{figure}[h!]
\begin{center}
\includegraphics[width=420px]{img/cal_mono.png}
\caption{Setup per la calibrazione del monocromatore.}
\end{center}
\end{figure}
\\Dopo aver allineato il setup sperimentale, si è collegata la fibra ottica al collettore avvitato all'uscita del monocromatore. Accesa la lampada alogena, si sono presi i picchi in uscita per 26 lunghezze d'onda differenti, distanziate fra di loro di 25 unità del monocromatore (1 u $\sim$ 1 nm). Dalle misure prese si sono effettuati fit gaussiani e si è estratta la legge di conversione in lunghezza d'onda.
\begin{figure}[!h]
\centering
    \subfloat[]{{\includegraphics[width=7.9cm]{img/spec_mono.pdf} }}%
    \subfloat[]{{\includegraphics[width=7.9cm]{img/cal_mono.pdf} }}%
\caption{Fit gaussiani dei picchi (a) e calibrazione del monocromatore (b).}
\end{figure}
\\
\section{Misura della fotoconducibilità dei semiconduttori}
Per la misure di fotocorrente e trasmittanza, si è installato un fotodiodo in uscita dal monocromatore,  distanziato in modo da riuscire ad alloggiare i campioni. Prima di effettuare qualunque misura, si è preso lo spettro di riferimento del setup, stabilendo così l'intensità di riferimento.\\
L'inserimento del chopper, il cui scopo è la modulazione del fascio ad una frequenza definita, non è necessario per questa serie di misure, ma ci permette di essere consistenti con le successive che utilizzano l'amplificatore lock-in.
\begin{figure}[h!]
\begin{center}
\includegraphics[width=400px]{img/foto_rif.png}
\caption{Setup per la misura dell'intensità di riferimento.}
\end{center}
\end{figure}
\\Per verificare che il vetrino su cui sono applicati i campioni fosse ininfluente, si è ripetuta la misura inserendo un vetrino da 920 $\mu$m all'uscita del monocromatore. Le misure prese sono risultate compatibili, per cui si è determinato che il vetrino non è significativo nel setup.\\
Le misure di fondo sono state prese alla sostituzione del campione, in modo da ridurre i problemi derivati da modifiche involontarie al setup.
\begin{figure}[h!]
\begin{center}
\includegraphics[width=350px]{img/fondo.pdf}
\caption{Misure di fondo, il colore è relativo al campione per cui sono state utilizzate.}
\end{center}
\end{figure}
\subsection{Seleniuro di Gallio - Oscilloscopio}
\begin{figure}[h!]
\begin{center}
\includegraphics[width=380px]{img/foto_osc.png}
\caption{Setup per la misura della fotocorrente e trasmittanza con l'oscilloscopio.}
\end{center}
\end{figure}
Inserito il campione di semiconduttore da caratterizzare, si è connessa una tensione di bias di 15 Vdc al campione, mentre l'uscita è collegata all'ingresso dell'oscilloscopio (all'interno del campione è presente un resistore da 1 M$\Omega$ per convertire la fotocorrente in tensione). Per ogni lunghezza d'onda, vengono misurate fotocorrente e intensità della luce trasmessa.
\begin{figure}[!h]
\centering
    \subfloat[]{{\includegraphics[width=7.9cm]{img/gase_osc_foto.pdf} }}%
    \subfloat[]{{\includegraphics[width=7.9cm]{img/gase_osc_trasm.pdf} }}%
\caption{Fit della fotocorrente e della trasmittanza per il campione GaSe.}
\end{figure}
Le misure della fotocorrente nella zona di assorbimento dovrebbero essere teoricamente costanti, ma nel nostro caso, dato che la sorgente non risulta ideale e che non si è effettuata una correzione dei dati, questi dipendono dallo spettro emesso dalla lampada. Per la trasmittanza invece, si è valutato il rapporto tra campione e fondo, eliminando così le variazioni di intensità luminosa.\\
Le lunghezze d'onda di soglia estratte dai fit sono compatibili, quindi il valore ponderato è $\left(601\pm5\right)$ nm e l'energy gap associato pari a $\left(2.06\pm0.01\right)$ eV. Il valore atteso di energy gap è di 2.1 eV, quindi la misura non è compatibile.
\newpage
\subsection{Seleniuro di Gallio - Lock-in}
La procedura di misurazione precedente è stata svolta nuovamente utilizzando l'amplificatore lock-in. Questo dispositivo, utilizzando il segnale di riferimento proveniente dal chopper, permette di amplificare il segnale con la medesima frequenza, riducendo così il rumore del resto dello spettro. Questo è necessario in quanto l'oscilloscopio era al limite di risoluzione, in quanto la fotocorrente è molto bassa. Per aumentare ulteriormente la risoluzione della misura, l'uscita dell'amplificatore è stata collegata all'oscilloscopio e misurata utilizzando cursori e funzioni matematiche integrate. L'errore sulle misure di tensione è stato determinato dalla deviazione standard calcolata dall'oscilloscopio su 100 campionamenti.
\begin{figure}[h!]
\begin{center}
\includegraphics[width=380px]{img/foto_lock.png}
\caption{Setup per la misura della fotocorrente e trasmittanza con l'amplificatore lock-in.}
\end{center}
\end{figure}
\begin{figure}[!h]
\centering
    \subfloat[]{{\includegraphics[width=7.8cm]{img/gase_lock_foto.pdf} }}%
    \subfloat[]{{\includegraphics[width=7.8cm]{img/gase_lock_trasm.pdf} }}%
\caption{Fit della fotocorrente e della trasmittanza per il campione GaSe.}
\end{figure}
\\Le lunghezze d'onda estratte dai fit sono compatibili fra di loro e il valor medio è $\left(603\pm5\right)$ nm, con un energy gap di $\left(2.06\pm0.01\right)$ eV. L'energy gap corrisponde a quello misurato con l'oscilloscopio, quindi a sua volta non compatibile con il valore atteso. Non essendo a conoscenza dell'errore sul valore atteso, non è corretto rifiutare la misura, in quanto prossima al valore teorico. La discordanza può essere dovuta alla sorgente non uniforme e alla risoluzione di lunghezza d'onda del monocromatore non sufficiente.
\subsection{Arseniuro di Gallio - Lock-in}
Si è effettuato lo stesso procedimento con l'utilizzo del lock-in anche su ulteriori campioni. Le misure della fotocorrente non sono più state prese con l'oscilloscopio, in quanto si desiderava valutare quanto fosse significativa la precisione maggiore nella misura.
\begin{figure}[!h]
\centering
    \subfloat[]{{\includegraphics[width=7.9cm]{img/gaas_foto.pdf} }}%
    \subfloat[]{{\includegraphics[width=7.9cm]{img/gaas_trasm.pdf} }}%
\caption{Fit della fotocorrente e della trasmittanza per il campione GaAs.}
\end{figure}
\\I valori delle lunghezze d'onda estratti dai fit risultano compatibili fra di loro. Il valor medio da utilizzare per la compatibilità con il valore atteso è $\left(862\pm8\right)$ nm. Il band gap risulta essere $\left(1.44\pm0.01\right)$ eV, compatibile con il valore teorico di 1.441 eV a 300 K.
\subsection{Campione Sconosciuto - Lock-in}
Il terzo campione analizzato è un campione non noto, per cui la nostra analisi è svolta per risalire al materiale di cui è composto. A vista, il campione si presenta di colore giallo-verde e molto trasparente; questo sta ad indicare una probabile deposizione in alto vuoto. 
\begin{figure}[!h]
\centering
    \subfloat[]{{\includegraphics[width=7.9cm]{img/boh_foto.pdf} }}%
    \subfloat[]{{\includegraphics[width=7.9cm]{img/boh_trasm.pdf} }}%
\caption{Fit della fotocorrente e della trasmittanza per il campione sconosciuto.}
\end{figure}
\\La lunghezza d'onda determinata con la fotocorrente è $\left(479\pm5\right)$ nm, mentre quella con la trasmittanza è $\left(506\pm2\right)$ nm.
I parametri estratti dal fit sono incompatibili tra di loro. Come si può vedere dai grafici, la presa dati non comprende l'intera sigmoide. Nel fit della trasmittanza, benché il p-value sia del 100\%, il parametro legato all'escursione risulta essere il doppio di quello atteso, quindi la regressione non è accettabile. \\L'energy gap relativo ad entrambe le misure è $\left(2.59\pm0.02\right)$ eV e $\left(2.45\pm0.02\right)$ eV. Osservando le tabelle, il dato ricavato dalla trasmittanza è vicino al solfuro di cadmio (CdS).
\section*{Conclusioni}
L'esperienza ha dato esito positivo. La calibrazione del setup è stata rapida ed efficace pur utilizzando il metodo approssimato. Il software fornito ci ha permesso di verificare le stime fatte in loco con un'analisi approfondita svolta su ROOT. Il monocromatore ha dato alcune difficoltà nell'allineamento, poi risolte osservando attentamente il meccanismo interno.\\
Le misure di fotoconducibilità dei campioni a base di Gallio, per via delle fotocorrenti elevate, sono stati facili da misurare e hanno dato risultati compatibili con i valori attesi; questi campioni sono anche i più recenti e quindi non mostrano segni di deterioramento dovuto al tempo, questo è ben visibile nei grafici di trasmittanza in appendice B dove si vede un netto scalino e due plateau molto ben definiti. La resa dei campioni rimanenti è stata scarsa. A causa dei danni al campione di HgI$_2$ si è scelto di non misurare la fotocorrente, in quanto sarebbe stato particolarmente difficile ottenere un dato attendibile. Per il campione sconosciuto, le misure prese sono state insufficienti e caratterizzate da particolare rumore ed altri effetti, rendendo così le regressioni non accettabili; detto ciò, la lunghezza d'onda di soglia estratta dal fit sulla trasmittanza misurata con il fotodiodo è compatibile con quella estratta dal fit sui dati dello spettrometro, che a sua volta è compatibile con il valore atteso relativo al solfuro di cadmio. Ci sentiamo di dire quindi, con una significatività ridotta, che il materiale di questo campione sia appunto CdS.
\begin{center}
\adfopenflourishleft\quad\adfast3\quad\adfopenflourishright
\end{center}
\begin{thebibliography}{}
\bibitem{1} Picollo F. et al., \emph{Slides sugli esperimenti}, Campusnet, 2019.
\bibitem{2} Vari autori, \emph{Wikipedia}, 2019.
\bibitem{3} CODATA, \emph{Fundamental Physics Constants}, CODATA 2014.
\end{thebibliography}
\newpage
\appendix
\section{Misura di capacità con l'amplificatore lock-in}
Per impratichirsi con l'amplificatore lock-in, si è voluta misurare una capacità incognita costruendo un partitore di corrente ed eccitandolo con un segnale sinusoidale di 4 Vpp e frequenza variabile. Prima le misure si sono prese con l'oscilloscopio e successivamente utilizzando l'amplificatore. La scelta delle frequenze ha compreso la frequenza di rete e le sue armoniche, in modo da valutare se il lock-in fosse in grado di ridurre il rumore generato.
\begin{figure}[!h]
\centering
    \subfloat[]{{\includegraphics[width=7.8cm]{img/caposc.pdf} }}%
    \subfloat[]{{\includegraphics[width=7.8cm]{img/caplock.pdf} }}%
\caption{Fit della capacità misurata con oscilloscopio e amplificatore lock-in.}
\end{figure}
Le due regressioni sono compatibili fra di loro. La capacità misurata con l'oscilloscopio risulta $\left(1.5\pm0.2\right)$ nF, mentre quella determinata con il lock-in è $\left(1.54\pm0.05\right)$ nF. La misura del lock-in, che dovrebbe essere più precisa, risulta incompatibile con il valore atteso. Questo risultato non è inaspettato in quanto il setup sperimentale risultava avere molte capacità parassite, che possono spiegare la differenza rispetto al valore teorico. Per ridurre questo problema, non si sarebbe dovuta utilizzare una breadboard e cavi troppo lunghi. Inoltre come previsto, le misure alla frequenza di rete e prime armoniche risultano più rumorose, rispetto a quelle fatte a frequenze più alte.
\section{Misure della trasmittanza con lo spettrometro}
Per la misura della trasmittanza, si sono utilizzate le utility del software dello spettrometro, che disponevano già del calcolo della trasmittanza. Per utilizzarle, si è dovuto prendere uno spettro di riferimento della sorgente luminosa ed uno spettro scuro, comprendente il rumore di fondo. Successivamente attivata la modalità trasmittanza, si è acquisita la risposta dei campioni sullo spettro di riferimento. Per evitare di saturare e avere la miglior risoluzione, si è usata la fibra ottica senza collettore. Per ogni insieme di dati presi con lo spettroscopio si è effettuata una regressione con un fit sigmoidale, in modo da poterlo confrontare con i dati acquisiti durante l'esperienza.
\begin{figure}[!h]
\centering
    \subfloat[]{{\includegraphics[width=7.9cm]{img/TeFGaSeSpec.pdf} }}%
    \subfloat[]{{\includegraphics[width=7.9cm]{img/TeFGaAsSpec.pdf} }}%
    \\ \subfloat[]{{\includegraphics[width=7.9cm]{img/TeFBohSpec.pdf} }}%
	 \subfloat[]{{\includegraphics[width=7.9cm]{img/TeFHgISpec.pdf} }}%
\caption{Fit della trasmittanza misurata con lo spettrometro per i diversi campioni.}
\end{figure}
\\Come si può vedere dai grafici, i campioni di GaSe e GaAs hanno una risposta più netta, mentre gli altri due sono meno definiti; questo spiega l'invecchiamento dei campioni e come questo influenzi le proprietà del materiale misurate.
\section{Dati e regressioni}
\subsection{Calibrazione Spettrometro}
\subsection{Calibrazione Monocromatore}
\subsection{Misura di riferimento del fondo}
\subsection{Misure campione GaSe}
\subsection{Misure campione GaAs}
\subsection{Misure campione sconosciuto}
\subsection{Misure campione HgI$_2$}
\end{document}
